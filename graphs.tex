%!TEX root = alga.tex
\section{Graphs \`{a} la carte}\label{sec-a-la-carte}

In this section we define several useful \hs{Graph} instances, and
show that the algebra presented in the previous section~\S\ref{sec-algebra} is
not restricted only
to directed graphs, but can be extended to axiomatically represent undirected,
reflexive, transitively closed, and labelled graphs, their various
combinations, and even hypergraphs.

\begin{figure}[b]
\vspace{-3mm}
\begin{minted}{haskell}
import           Data.Set (Set, @\blk{singleton, union, elems, fromAscList}@)
import qualified Data.Set as Set (empty)
\end{minted}
\vspace{1mm}
\begin{minted}{haskell}
data Relation a = Relation { domain :: Set a, relation :: Set (a, a) } deriving (Eq, Show)
\end{minted}
\vspace{1mm}
\begin{minted}{haskell}
instance Ord a => Graph (Relation a) where
    type Vertex (Relation a) = a
    empty       = Relation Set.empty Set.empty
    vertex  x   = Relation (singleton x) Set.empty
    overlay x y = Relation (domain x `union` domain y) (relation x `union` relation y)
    connect x y = Relation (domain x `union` domain y) (relation x `union` relation y
        `union` fromAscList [ (a, b) | a <- elems (domain x), b <- elems (domain y) ])
\end{minted}
\vspace{-3mm}
\caption{Implementing the \hs{Graph} type class by a binary relation
and the core graph construction primitives
defined in~\S\ref{sub-constructing}\label{fig-relation}}
\vspace{-3mm}
\end{figure}

\subsection{Binary relation}\label{sub-relation}

We start by a direct encoding of the graph construction primitives defined
in~\S\ref{sub-constructing} into the \hs{Relation} data type isomorphic
to a pair of sets $(V,E)$, see Fig.~\ref{fig-relation}. As we have seen,
this implementation satisfies the axioms of the graph algebra. Furthermore, it
is a \emph{free graph} in the sense that it does not satisfy any other laws.
This follows from the fact that any algebraic graph expression $g$ can be
rewritten in the following \emph{canonical form}:
\[
g = \Big(\sum_{v\in V_g} v\Big) + \Big(\sum_{(u,v)\in E_g} \hspace{-1mm} u \rightarrow v\Big),
\]
\vspace{-1mm}
\noindent
where $V_g$ is the set of vertices that appear in $g$, and $(u,v)\in E_g$ if
vertices $u$ and $v$ appear in the left-hand and right-hand arguments of
the connect operation $\rightarrow$ at least once (and should thus be connected
by an edge). The canonical form of an
expression $g$ can be represented as \hs{Relation@$\,\,V_g\,E_g$@},
and any additional law on \hs{Relation} would violate the canonicity property.
The existence of the canonical form has been proved by~\citet{2014_algebra_mokhov}
for an extended version of the algebra. The proof fundamentally builds on the
decomposition axiom: one can apply it repeatedly to an expression, breaking up
connect sequences $x\rightarrow y\rightarrow z$ into pairs $x \rightarrow y$
until the decomposition can no longer be applied. We can then open the parentheses
using the distributivity and rearrange terms into the canonical expression
by the commutativity and idempotence of the overlay $+$ operation.

It is convenient to make \hs{Relation} an instance of the \hs{Num} type class
to use the standard $+$ and $*$ operators as shortcuts for \hs{overlay} and
\hs{connect}, respectively:

\begin{minted}{haskell}
instance (Ord a, Num a) => Num (Relation a) where
    fromInteger = vertex . fromInteger
    (+)         = overlay
    (*)         = connect
    signum      = const empty
    abs         = id
    negate      = id
\end{minted}

\noindent
Note that the \hs{Num} law \hs{@\blk{abs}@ x * signum x == x} is satisfied by the above
definition since $x \rightarrow \varepsilon = x$. In fact, any \hs{Graph} instance
can be made a law-abiding \hs{Num} instance if need be, using a definition similar to the above.

We can now play with graphs and binary relations using the interactive GHC:

\begin{minted}[frame=single]{haskell}
@\ghci@ 1 * (2 + 3) :: Relation Int
Relation {domain = fromList [1,2,3], relation = fromList [(1,2),(1,3)]}
@\ghci@ 1 * (2 + 3) + 2 * 3 == (clique [1..3] :: Relation Int)
True
@\ghci@ 1 * 2 == (2 * 1 :: Relation Int)
False
@\ghci \blk{:t}@ clique "abcd"
clique "abcd" :: (Graph g, Vertex g @\teq@ Char) => g
@\ghci @relation (clique "abcd")
fromList [('a','b'),('a','c'),('a','d'),('b','c'),('b','d'),('c','d')]
\end{minted}

\noindent
The last example highlights the fact that the \hs{Relation@\,\,\blk{a}@} instance allows vertices
of any type \hs{@\blk{a}@} that satisfies the~\hs{Ord@\,\,\blk{a}@} constraint.

\subsection{Deep embedding}\label{sub-embedding}

By embedding the core graph construction primitives into a basic algebraic data type
we obtain the following simple \hs{Graph} instance:

\begin{minted}{haskell}
data Expr a = Empty | Vertex a | Overlay (Expr a) (Expr a) | Connect (Expr a) (Expr a)
    deriving Show

instance Graph (Expr a) where
    type Vertex (Expr a) = a
    empty   = Empty
    vertex  = Vertex
    overlay = Overlay
    connect = Connect
\end{minted}

The instance definition provides a mapping from the \emph{shallow embedding}
of the core primitives, represented by the type class \hs{Graph}, into the
corresponding \emph{deep embedding}, represented by the data type \hs{Expr}.
It is known, e.g. see~\citet{2014_gibbons_folding}, that by \emph{folding} the data
type one can always obtain the inverse mapping:

\begin{minted}{haskell}
fold :: (Graph g, Vertex g @\teq@ a) => Expr a -> g
fold Empty         = empty
fold (Vertex  x  ) = vertex x
fold (Overlay x y) = overlay (fold x) (fold y)
fold (Connect x y) = connect (fold x) (fold y)
\end{minted}

We cannot use the derived \hs{Eq} instance of the \hs{Expr} data type, because it
would clearly violate the axioms of the algebra, e.g. \hs{Overlay@\,\,@Empty@\,\,@Empty} is
structurally different from \hs{Empty}, but they must be equal according to the axioms.
However, we can implement a custom law-abiding \hs{Eq}
instance by \emph{reinterpreting} the graph expression \hs{Expr}
as a \hs{Relation}, thereby gaining access to the canonical graph representation:

\begin{minted}{haskell}
instance Ord a => Eq (Expr a) where
    x == y = fold x == (fold y :: Relation a)
\end{minted}

An interesting feature of the \hs{Expr} instance is that it allows to represent
densely connected graphs more compactly. For example, \hs{clique [1..n] :: Expr Int}
has a linear-size representation in memory, while \hs{clique [1..n] :: Relation Int}
stores each edge separately and therefore uses $O(n^2)$ memory. Exploiting the
compact graph representation for deriving algorithms that are asymptotically faster
on dense graphs, compared to conventional algorithms operating on `uncompressed'
graph representations isomorphic to $(V,E)$, is outside the scope of this paper,
but is an interesting direction of future research.

\begin{figure}[b]
\vspace{-2mm}
\begin{minted}{haskell}
import           Data.Map.Strict (Map, keysSet, fromSet)
import qualified Data.Map.Strict as Map
import           Data.Set (Set)
import qualified Data.Set as Set
\end{minted}
\vspace{1mm}
\begin{minted}{haskell}
newtype AdjacencyMap a = AM { adjacencyMap :: Map a@\,\,@(Set a) }@\,\,@deriving@\,\,@(Arbitrary,@\,\,@Eq,@\,\,@Show)
\end{minted}
\vspace{1mm}
\begin{minted}{haskell}
instance Ord a => Graph (AdjacencyMap a) where
    type Vertex (AdjacencyMap a) = a
    empty       = AM $ Map.empty
    vertex  x   = AM $ Map.singleton x Set.empty
    overlay x y = AM $ Map.unionWith Set.union (adjacencyMap x) (adjacencyMap y)
    connect x y = AM $ Map.unionsWith Set.union [ adjacencyMap x, adjacencyMap y,
        fromSet (const . keysSet $ adjacencyMap y) (keysSet $ adjacencyMap x) ]
\end{minted}
\vspace{-3mm}
\caption{Implementing the \hs{Graph} type class by adjacency map\label{fig-adjacency-map}}
\vspace{-1mm}
\end{figure}

\subsection{Adjacency map and topological sort}\label{sub-adjacency-map}

In this subsection we show how to reuse existing graph libraries, such as
\textsf{containers}, by wrapping them into the algebraic graph API. More specifically,
we reuse the efficient implementation of the \emph{topological sort} developed
by~\citet{1995_king_graphs}.

The interface of the \textsf{containers} library uses the adjacency list
representation of graphs and we therefore define a new \hs{Graph} instance that
provides the closest match, see Fig.~\ref{fig-adjacency-map}. In our experiments
this instance is often faster than \hs{Relation} because of more efficient
sharing of common subgraphs in memory, see~\S\ref{sub-benchmarks}.

The \hs{AdjacencyMap@\,\,\blk{a}@} instance is parametric and in order to use the
\hs{topSort} function provided by \textsf{containers}, we need to map the graph vertices
into a contiguous subsets of integers \hs{Int}. This can be done by the function
\hs{graphFromEdges} defined in \textsf{containers} that requires an adjacency list as
the input. We can obtain the adjacency list from \hs{AdjacencyMap@\,\,\blk{a}@} as follows:

\begin{minted}{haskell}
adjacencyList :: AdjacencyMap a -> [(a, [a])]
adjacencyList = @\std{map}@ (@\std{fmap}@ Set.toAscList) . Map.toAscList . adjacencyMap
\end{minted}

\noindent
Let's test that \hs{AdjacencyMap@\,\,\blk{a}@} is a valid \hs{Graph} instance and
demonstrate \hs{adjacencyList}:

\begin{minted}[frame=single]{haskell}
@\ghci @quickCheck (axioms :: GraphTestsuite (AdjacencyMap Int))
@\blk{+++ OK, passed 100 tests.}@

@\ghci @adjacencyList $ clique [1..4]
[(1,[2,3,4]),(2,[3,4]),(3,[4]),(4,[@@])]
\end{minted}


We do not show the details of our
implementation here, as they
