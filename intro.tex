%!TEX root = alga.tex
\vspace{-2mm}
\section{Introduction}\label{sec-intro}

Graphs are ubiquitous in computing, yet working with graphs often requires
painfully low-level fiddling with sets of vertices and edges. Building high-level
abstractions is difficult, because the commonly used foundation -- the pair $(V, E)$
of vertex set $V$ and edge set $E \subseteq V \times V$ -- is a source of partial
functions. We can represent the pair by the following simple data type\footnote{Although
in this paper we exclusively use Haskell, the problem we solve is general and the
proposed approach can be readily adapted to other programming languages.}:

\vspace{0.5mm}
\begin{minted}{haskell}
  data G a = G { vertices :: [a], edges :: [(a,a)] }
\end{minted}
\vspace{0.5mm}

\noindent
Now \hs{G [1,2,3] [(1,2),(2,3)]} is the graph with three vertices $V = \{1,2,3\}$ and
two edges $E = \{(1,2), (2,3)\}$. The consistency invariant $E \subseteq V \times V$ holds.
But what is \hs{G [1] [(1,2)]}? The edge refers to the non-existent vertex $2$, breaking the
invariant, and there is no easy way to reflect this in types. Perhaps, our data type is just
too simplistic; let us look at state-of-the-art graph libraries instead.

The \textsf{containers} library is designed for performance and powers GHC itself. It
represents graphs by \emph{adjacency arrays}~\cite{1995_king_graphs} that are not
better than our data type from the safety point of view: the \hs{buildG} graph construction
function is partial and can fail with the \textsf{`index out of range'} error.
Another popular library \textsf{fgl} uses the \emph{inductive graph
representation}~\cite{2001_erwig_inductive}, but its graph construction API has similar
partial functions, e.g. \hs{insEdge} can fail with the \textsf{`edge from non-existent
vertex'} error.

Both \textsf{containers} and \textsf{fgl} are treasure troves of graph algorithms,
but it is easy to make an error when using them. Is there a safe graph construction
interface we can build on top?

In this paper we present \emph{algebraic graphs} --- a new interface
for graph construction and transformation. We abstract away from graph representation
details and characterise graphs by a set of axioms, much like numbers are algebraically
characterised by \emph{rings}~\cite{1999_maclane_algebra}. Our approach is based on
the \emph{algebra of parameterised graphs}, a mathematical formalism used in digital
circuit design~\cite{2014_algebra_mokhov}, which we simplify and adapt to the context
of functional programming.

Algebraic graphs have a safe and minimalistic core of four graph construction primitives,
as captured by the following data type:

\begin{minted}{haskell}
data Graph a = Empty
             | Vertex a
             | Overlay (Graph a) (Graph a)
             | Connect (Graph a) (Graph a)
\end{minted}

\noindent
Here \hs{Empty} and \hs{Vertex} construct the \emph{empty} and \emph{single-vertex} graphs,
respectively; \hs{Overlay} composes two graphs by taking the union of their vertices and
edges, and \hs{Connect} is similar to \hs{Overlay} but also creates edges between vertices
of the two graphs, see~Fig.~\ref{fig-construction} for examples. The \emph{overlay} and
\emph{connect} operations have two important properties:
i) they are closed on the set of graphs, i.e. are total functions, and ii) they can be used
to construct any graph starting from the empty and single-vertex graphs.
For example, \hs{Connect (Vertex 1) (Vertex 2)} is the graph with two vertices $\{1,2\}$
and a single edge $(1,2)$. Malformed graphs, such as \hs{G [1] [(1,2)]}, cannot be
expressed in this core language.

The main goal of this paper is to demonstrate that \emph{this core is a safe, flexible
and elegant foundation for working with graphs}. Our specific contributions are:
\begin{itemize}
  \item Compared to existing libraries, algebraic graphs have a smaller
  core (just four graph construction primitives), are more compositional
  (hence greater code reuse), and have no partial functions (hence fewer
  opportunities for usage errors). We present the core and justify these claims
  in \S\ref{sec-core}.
  \vspace{0.5mm}

  \item The core has a simple mathematical structure fully characterised
  by a set of axioms~(\S\ref{sec-algebra}). This makes the
  proposed interface easier for testing and formal verification. We show that
  the core is \emph{complete}, i.e. any graph can be constructed, and \emph{sound},
  i.e. malformed graphs cannot be constructed.
  \vspace{0.5mm}

  \item Under the basic set of axioms, algebraic graphs correspond to directed
  graphs with no edge labels. As we show in~\S\ref{sec-a-la-carte}, by extending
  the algebra   with additional axioms, it is possible to also represent undirected,
  reflexive and transitive graphs, their combinations, as well as hypergraphs.
  Importantly, the core remains unchanged, which allows us to define highly
  reusable polymorphic functions on graphs.
  \vspace{0.5mm}

  \item We develop a library\footnote{The library is on Hackage:
  \url{http://hackage.haskell.org/package/algebraic-graphs}.}
  for constructing and transforming algebraic graphs and demonstrate its
  flexibility in \S\ref{sec-transformations}.
  % Although the development of efficient algorithms for algebraic
  % graphs is outside the scope of this paper, we show that the library can cope
  % with graphs comprising billions of edges in the matter of
  % seconds, which is sufficiently fast for many applications.
\end{itemize}

Graphs and functional programming have a long history. We review related
work in \S\ref{sec-related}. Limitations of the presented approach and future
research directions are discussed in \S\ref{sec-discussion}.
