%!TEX root = alga.tex
\section{Introduction}

Graphs are ubiquitous in computing, yet working with graphs often requires
painfully low-level fiddling with sets of vertices and edges. We highlight this
in the motivation section~\S\ref{sec-motivation} by examining the APIs of
two state-of-the-art Haskell graph libraries\footnote{In this paper we exclusively
use Haskell, but the presented approach can be adapted to other functional
programming languages.}. To address this problem, we present
\emph{algebraic graphs} --- a new interface for graph construction and
transformation. Our specific contributions are:

\begin{itemize}
  \item Compared to existing libraries, algebraic graphs have a smaller
  \emph{core} (just four graph construction primitives), are more compositional
  (hence greater code reuse), and have no partial functions (hence fewer
  opportunities for usage errors). We present the core and justify these claims
  in \S\ref{sec-core}.

  \item The core has a simple mathematical structure and is fully characterised
  by a set of algebraic laws described in~\S\ref{sec-algebra}, which makes the
  proposed interface easier for testing and formal verification. We show that
  the core is \emph{complete}, i.e. any graph can be constructed, and \emph{sound},
  i.e. malformed graphs cannot be constructed.

  \item Under the basic set of axioms, algebraic graphs correspond to unlabelled
  directed graphs. As we show in~\S\ref{sec-a-la-carte}, by extending the algebra
  with additional axioms, it is possible to also represent undirected, reflexive,
  transitively closed, and labelled graphs. Importantly, the core graph
  construction primitives remain unchanged, which allows to define highly
  reusable polymorphic functions on graphs.

  \item We develop a library of polymorphic graph construction and
  transformation functions on top of the algebraic core and demonstrate its
  flexibility in \S\ref{sec-transformations}.
  Although the development of asymptotically optimal algorithms for algebraic
  graphs is outside the scope of this paper, we show that the library can cope
  with graphs comprising millions of vertices and billions of edges in the
  matter of seconds, which is sufficiently fast for many real-life applications.
  We give two examples of such applications in \S\ref{sec-applications}.
\end{itemize}

Graphs and functional programming have a long history. We review related
work in \S\ref{sec-related}. Limitations of algebraic graphs and future
research directions are discussed in \S\ref{sec-discussion}.

\section{Motivation}\label{sec-motivation}

As a motivating
example, we look at two popular Haskell graph libraries. The first one is
available from the \textsf{containers} package and uses adjacency arrays
to represent graphs,
as described by~\citet{1995_king_graphs}. The second one
is the \textsf{fgl} library, which is based on \emph{inductive graphs}
by~\citet{2001_erwig_inductive}. Both libraries are
actively maintained\footnote{As of writing, the latest versions of the libraries
are \textsf{containers-0.5.10.1} and \textsf{fgl-5.5.3.0}.},
have many users, and provide efficient implementations of
fundamental graph algorithms. We argue, however, that using these libraries
can be difficult and error-prone, and show how one can access their algorithmic
power via a simple and safe interface based on algebraic graphs.

\begin{figure}
\begin{subfigure}[b]{0.28\linewidth}
\begin{minted}[fontsize=\small]{haskell}
import Data.Graph

-- Create graph 1 --> 2
a :: Graph
a = buildG (1,2) [(1,2)]

-- Add edge 3 --> 1 to a
b :: Graph
b = buildG (1,3) $
    edges a ++ [(3,1)]

-- Remove vertex 2 from b
c :: Graph
c = buildG (1,3) $ filter
    (notAdjacent 2) (edges b)
\end{minted}
\caption{\textsf{containers} library}
\end{subfigure}
\hfill
\hfill
\vrule
\hfill
\hfill
\begin{subfigure}[b]{0.3\linewidth}
\begin{minted}[fontsize=\small]{haskell}
import Data.Graph.Inductive

-- Create graph 1 --> 2
a :: Graph gr => gr () ()
a = mkUGraph [1,2] [(1,2)]

-- Add edge 3 --> 1 to a
b :: DynGraph gr => gr () ()
b = insEdge (3,1,()) $
    insNode (3,()) a

-- Remove vertex 2 from b
c :: DynGraph gr => gr () ()
c = delNode 2 b
@ @
\end{minted}
\caption{\textsf{fgl} library}
\end{subfigure}
\hfill
\vrule
\hfill
\hfill
\begin{subfigure}[b]{0.35\linewidth}
\begin{minted}[fontsize=\small,escapeinside=@@]{haskell}
import Algebra.Graph

-- Create graph 1 --> 2
a :: (Graph g, Vertex g @\teq@ Int) => g
a = connect (vertex 1) (vertex 2)

-- Add edge 3 --> 1 to a
b :: (Graph g, Vertex g @\teq@ Int) => g
b = overlay a $
    connect (vertex 3) (vertex 1)

-- Remove vertex 2 from b
c :: (Graph g, Vertex g @\teq@ Int) => g
c = removeVertex 2 b
@ @
\end{minted}
\caption{Algebraic graphs}
\end{subfigure}
\caption{Simple transformations of unlabelled directed graphs\label{fig-example}}
\vspace{-4mm}
\end{figure}

Fig.~\ref{fig-example} compares the \textsf{containers} and
\textsf{fgl} libraries with algebraic graphs on a sequence of
simple graph transformations: we start with graph $a$ containing the
edge $1 \rightarrow 2$; graph $b$ is obtained by adding the vertex $3$
and the edge $3 \rightarrow 1$ to $a$; finally, graph $c$ is obtained by removing
the vertex $2$ from graph $b$.

Fig.~\ref{fig-example}(a) shows how to realise these graph transformations
using the \textsf{containers} library. To construct graph~$a$ one needs
to specify both the set of vertices and edges. Note that
\hs{buildG} is \emph{partial} and will fail if these sets are inconsistent.
As an example, \hs{buildG (2,2) [(1,2)]} is well-typed but fails with the
\textsf{`index out of range'} error, since the edge refers to the non-existent vertex 1.
To construct $b$, we extend the vertex set and append the
edge $3 \rightarrow 1$ to the list of edges of $a$; the \textsf{containers} library does
not support sharing of common subgraphs and new memory is allocated for
storing the whole graph $b$. Furthermore, the library is
\emph{incomplete}: the graphs whose vertex sets are not contiguous subsets of
integers cannot be represented. Graph $c$ is an example: its vertex set is
$\{1,3\}$ and the best we can do is to remove all edges adjacent to vertex 2
leaving it isolated in the resulting graph.

The \textsf{fgl} library defines the type class \hs{Graph} and its subclass
\hs{DynGraph} for working with static (unchangeable) and dynamic (changeable)
graphs, comprising 10 class methods in total. As shown in Fig.~\ref{fig-example}(b),
to create graph~$a$ we call the \hs{mkUGraph} method, which is similar to
\hs{buildG} from the \textsf{containers} library and also needs to deal with the
awkward case when the sets of vertices and edges are inconsistent; instead of
failing with an error, \textsf{fgl} chooses to ignore edges adjacent to non-existent
vertices, hence \hs{mkUGraph [}\hs{] [(1,2)]} creates the empty graph. To construct
graph~$b$, we first add the vertex 3 to $a$ and then add the edge $3 \rightarrow 1$.
If we skip the first step, \hs{insEdge} fails with the
\textsf{`edge from non-existent vertex'} error.
Note that \textsf{fgl} is optimised for working with labelled graphs, which requires
using the dummy label \hs{()} for vertices and edges when working with unlabelled
graphs. Graph~$c$ is obtained in a straightforward way by calling the \hs{delNode}
function. As one can see, the resulting expression is still polymorphic, admitting
multiple possible implementations. The default implementation of the \hs{DynGraph}
class supports subgraph sharing, thus being more memory efficient compared to
the \textsf{containers} library.

Algebraic graphs are built on the minimalistic type class \hs{Graph} comprising
only four methods \hs{empty}, \hs{vertex}, \hs{overlay} and \hs{connect}, as
defined in \S\ref{sec-core}. The code in
Fig.~\ref{fig-example}(c) intentionally avoids helper library functions (apart
from \hs{removeVertex}), performing the intended graph transformations using the
bare-metal primitives of the algebra. Graph~$a$ is constructed by \emph{connecting}
vertices 1 and 2. This definition has no hidden partiality or ambiguity.
Unlike the \textsf{containers} and \textsf{fgl} libraries, the type of vertices
is not fixed to \hs{Int}, e.g. we could have used \hs{Integer} instead.
Graph~$b$ is constructed by \emph{overlaying} $a$ with the edge $3 \rightarrow 1$.
To remove the vertex 2, we use the polymorphic library function
\hs{removeVertex} that essentially replaces the vertex with the \hs{empty} graph,
as explained in \S\ref{sec-transformations}. We argue that the algebraic
interface is superior, because it abstracts away such details as sets of
vertices and edges, eliminating the need for partial functions.

The main idea of this paper is that the proposed minimalistic core of four graph
construction primitives provides a safe and flexible algebraic interface
for working with graphs, and also unlocks a new world of graph instances, with
classic inhabitants from \textsf{containers} and \textsf{fgl}, as well
as new, unknown forms of life.
