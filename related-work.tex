%!TEX root = alga.tex
\section{Related Work}\label{sec-related}

Historically, first approaches to graph representation in functional programming
used edge lists, adjacency lists, as well as mutually recursive data structures
representing cyclic graphs by the so-called `tying the knot' approach. The former
were generally slower
than their imperative counterparts, while the latter were very difficult to
work with. An asymptotically optimal implementation of the depth-first search
algorithm developed by~\citet{1995_king_graphs} used arrays to represent graphs
and \emph{state-transformer monads}~\cite{1994_launchbury_st} to mimic imperative array
updates in pure functional programming. The developed algorithms are still in
use today and are available from the \textsf{containers} library shipped with GHC.
The API of the library contains partial functions.

A fundamentally different approach by~\citet{2001_erwig_inductive} is based
on \emph{inductive graphs}, whereby a graph can be decomposed into a \emph{context}
(a node with its neighbourhood) and the rest of the graph. This inductive
definition makes it possible to share common subgraphs and provides a way to
implement graph algorithms in a more functional style compared to the previous
approaches based on array representations. Inductive graphs are implemented in
the \textsf{fgl} library that contains implementations of many standard graph
algorithms, from depth-first search to maximum flow on weighted graphs. The library
defines type classes \hs{Graph} and \hs{DynGraph} for working with static
(unchangeable) and dynamic (changeable) graphs, comprising 10 class methods in total.
Compared to algebraic graphs proposed in this paper, \textsf{fgl} has a larger
core of graph construction primitives (10 vs 4), some of which are partial. An
important advantage of \textsf{fgl} is the support of edge-labelled graphs.

Several other authors investigated ways to define graphs compositionally,
e.g.~\citet{1995_gibbons_algebra} proposed an algebraic framework for modelling
\emph{directed acyclic graphs} comprising 6 core graph construction primitives,
but the approach was not general enough to handle other practically useful classes
of graphs.

Gibbons's algebra is an example of a large body of research on \emph{categorical
graph algebras}, e.g. see a survey by~\citet{2010_selinger_survey}. These algebras
are typically much more complex than the one presented in this paper\footnote{As an
example, \emph{Signal Flow Graphs}~\cite{2015_bonchi_sfg} have 17 primitives
and a few dozens of laws. Smaller characterisations of Signal Flow Graphs exist,
however minimising the number of graph construction primitives has not (so far) been
a priority for the authors (private communication with Pawel Sobocinski).},
because they can represent graphs with
\emph{heterogeneous vertices and edges}, where not all vertices
and edges are \emph{compatible}. Graphs in this paper are \emph{homogeneous}, i.e.
an edge is allowed between any pair of vertices. This is a limitation for some
applications, but it allows us to have a much simpler theory and implementation.
\emph{Petri nets}~\cite{1989_murata_pn} is an example of graphs where not all
edges are allowed\footnote{Petri nets have vertices of two types, called \emph{places}
and \emph{transitions}, and edges are only allowed between vertices of different
types.}.
Algebraic graphs proposed in this paper cannot represent Petri
nets in a safe way.

From a very different angle, simple algebraic structures, such as \emph{semirings},
have been successfully applied to solving various path problems on graphs using
functional programming, e.g. see~\citet{2013_semirings_dolan}. These approaches
typically use matrix-based data structures for manipulating connectivity and distance
information with the goal of solving optimisation problems on graphs, and are not
suitable as an abstract interface for graph representation.

Simple graph construction cores are known for special families of graphs. For example,
non-empty \emph{series-parallel graphs} require only three primitives: a single vertex,
and series and parallel composition operations. A classical result~\cite{1979_valdes_sp}
states that only \emph{$N$-free graphs} can be constructed using these primitives.
Similarly, the family of \emph{cographs} corresponds to \emph{$P_4$-free graphs}, which
also require only three graph construction primitives: a single vertex, graph complement,
and disjoint graph union~\cite{1981_corneil_cographs}. Interestingly, there is an
alternative core for cographs: a single vertex, disjoint graph union, and disjoint graph
join. The only difference from the core used in this paper is the disjointness requirement.
By dropping this requirement, we can construct arbitrary graphs. In particular, both
$N = 1 \rightarrow 2 + 3 \rightarrow (2 + 4)$ and
$P_4 = 1 \rightarrow 2 + 2 \rightarrow 3 + 3 \rightarrow 4$ can be easily constructed.

This paper builds on the work by~\citet{2014_algebra_mokhov}, where
the \emph{algebra of parameterised graphs}, a mathematical
structure very similar to a semiring, was proposed as a complete and sound formalism
for graph representation in the context of digital circuit design. In that paper the
authors did not investigate applications of the algebra in functional programming but
proved many important results that are essential for this work.
\citet{2014_alekseyev_phd} derived a formalisation of the algebra of parameterised
graphs in Agda, using an encoding similar to the core type class that we define.

% \begin{itemize}
%     \item Structured Graphs by Oliveira & Cook. \url{https://www.cs.utexas.edu/~wcook/Drafts/2012/graphs.pdf}
%     \item "Algebras for graphs have been studied in the context of graph rewriting, see Bauderon and Courcelle (1986), for example."
% \end{itemize}
