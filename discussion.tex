%!TEX root = alga.tex
\section{Discussion, limitations and future research}\label{sec-discussion}

The paper presented \emph{algebraic graphs} --- a new approach to representing and
working with graphs in functional programming languages.
Compared to the state-of-the-art, algebraic graphs are easier to use,
more compositional, and have a smaller core of only four graph
construction primitives characterised by an elegant algebra of graphs.

We demonstrated the flexibility of algebraic graphs by numerous examples and
developed a library for polymorphic graph construction and transformation.

The presented approach has a few limitations:

\begin{itemize}
    \item There are no known efficient implementations of fundamental graph
    algorithms, such as depth-first search, that work directly on the algebraic
    core. Therefore, we need to translate core expressions to conventional
    graph representations, such as adjacency lists, and utilise existing graph
    libraries, which may be suboptimal for certain algorithmic problems.

    \item The presented \hs{Graph} instances incur a logarithmic overhead
    during graph construction and therefore do not scale well. The performance
    figures reported in~\S\ref{sub-library-summary} are not acceptable for
    high-performance applications. A promising direction to overcoming this limitation
    is to employ the \emph{discrimination} Haskell library that provides linear-time
    sorting and grouping algorithms~\cite{2012_henglein_discriminations}.

    \item This paper has not addressed labelled graphs. In particular, there is
    no algebraic characterisation of
    graphs with arbitrary vertex and edge labels. However,~\citet{2014_algebra_mokhov}
    give an algebraic characterisation for graphs labelled with Boolean functions.
    % \item The graph transformation library does not provide the functionality
    % for overriding the default implementation of functions that can be implemented
    % more efficiently by a specific data structure. An example of a function that
    % would particular benefit from overriding is \hs{removeEdge}, as defined
    % in~\S\ref{sub-beyond}.
\end{itemize}

Despite these limitations, algebraic graphs have been successfully used
in the design of asynchronous circuits~\cite{2015_beaumont_concepts} and
processor microcontrollers~\cite{2014_algebra_mokhov}.

Our future research will focus on addressing the limitations, as well as on
exploration of the following topics:

\begin{itemize}
    \item By using the algebraic approach to graph representation one can
    formulate graph algorithms in the form of finding a solution of an algebraic
    equation with unknowns.
    This may potentially open way to the discovery of novel graph algorithms.
    \item Algebraic graph expressions can be minimised via the
    \emph{modular decomposition} of graphs~\cite{2005_mcconnell_modular}, thereby
    reducing their memory footprint, as well as speeding up their processing.
    Exploiting the compactness of algebraic graphs in algorithms is a
    promising research direction.
\end{itemize}
