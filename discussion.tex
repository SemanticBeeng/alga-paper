%!TEX root = alga.tex
\vspace{-1mm}
\section{Discussion and future research opportunities}\label{sec-discussion}

The paper presented a new algebraic foundation for working with graphs.
It is particularly well-suited for functional programming languages
and benefits from functional programming abstractions, such as functors and
monads. Compared to the state-of-the-art, algebraic graphs are easier to use and reuse,
more compositional, and have a smaller core of only four graph
construction primitives, fully characterised by an elegant algebra of graphs.

We demonstrated the flexibility of algebraic graphs by several examples and
developed a Haskell library for constructing and transforming polymorphic graphs.

The presented approach has a few important limitations:

\begin{itemize}
    \item This paper has not addressed edge-labelled graphs. In particular, there is
    no known extension of the presented algebra characterising graphs with arbitrary
    vertex and edge labels.
    However,~\citet{2014_algebra_mokhov} give an algebraic characterisation for graphs
    labelled with Boolean functions, which can be generalised to labels that form
    a semiring.

    We found that one can represent edge-labelled graphs by functions from labels
    to graphs. For example, a \emph{finite automaton} can be thought of as a collection
    of graphs, one for each symbol of the alphabet:

    \vspace{2mm}
    \begin{minted}{haskell}
    type Automaton a s = a -> Relation s
    \end{minted}
    \vspace{2mm}

    \noindent
    Here \hs{a} and \hs{s} stand for the alphabet and the set of states of the
    automaton, respectively. This representation of labelled graphs is supported
    by the following graph instance:

    \vspace{2mm}
    \begin{minted}{haskell}
    instance Graph g => Graph (a -> g) where
        type Vertex (a -> g) = Vertex g
        empty       = pure empty
        vertex      = pure . vertex
        overlay x y = overlay <$> x <*> y
        connect x y = connect <$> x <*> y
    \end{minted}
    \vspace{2mm}

    \noindent
    Therefore, \hs{Automaton@\,\,\blk{a}\,\,\blk{s}@} is a valid \hs{Graph} instance.

    \vspace{0.5mm}
    \item As mentioned in~\S\ref{sec-related}, the presented approach is designed
    for homogeneous graphs, where an edge is allowed between any pair of vertices.
    It is an open research question whether it is possible to extend algebraic graphs
    for modelling heterogeneous graphs, such as Petri nets, without sacrificing the
    simplicity of the algebraic core.

    \vspace{0.5mm}
    \item Many graph instances, e.g. \hs{Relation}, incur a logarithmic overhead
    during graph construction, and may therefore be unsuitable for
    high-performance applications. One possible solution is to operate on
    deeply-embedded algebraic graphs (such as \hs{data Graph}), and perform
    conversions to more conventional representations only when necessary.

    \vspace{0.5mm}
    \item There are no known efficient implementations of fundamental graph
    algorithms, such as depth-first search, that work directly on the algebraic
    core. Therefore, we need to translate core expressions to conventional
    graph representations, such as adjacency lists, and utilise existing graph
    libraries, which may be suboptimal for certain algorithmic problems.
\end{itemize}

Despite these limitations, algebraic graphs have been successfully used
in the design of processor microcontrollers~\cite{2014_algebra_mokhov} and
asynchronous circuits~\cite{2015_beaumont_concepts}.

Our future research will focus on addressing the above limitations, and on the
exploration of the following topics:

\begin{itemize}
    \item Algebraic graph expressions can be minimised via the
    \emph{modular decomposition} of graphs~\cite{2005_mcconnell_modular}, thereby
    reducing their memory footprint, as well as speeding up their processing.
    Modular decomposition is a canonical graph representation, which can therefore
    be used to efficiently compare algebraic graph expressions for equality.
    Exploiting the compactness of algebraic graphs in algorithms is a
    promising research direction.
    \item By using the algebraic approach to graph representation one can
    formulate graph algorithms in the form of solving systems of algebraic
    equations with unknowns.
    This may potentially open way to the discovery of novel graph algorithms.
\end{itemize}
