%% For double-blind review submission
\documentclass[acmlarge,anonymous]{acmart}\settopmatter{printfolios=true}
%% For single-blind review submission
%\documentclass[acmlarge,review]{acmart}\settopmatter{printfolios=true}
%% For final camera-ready submission
%\documentclass[acmlarge]{acmart}\settopmatter{}

%% Some recommended packages.
\usepackage{booktabs}   %% For formal tables:
                        %% http://ctan.org/pkg/booktabs
\usepackage{subcaption} %% For complex figures with subfigures/subcaptions
                        %% http://ctan.org/pkg/subcaption

\makeatletter\if@ACM@journal\makeatother
%% Journal information (used by PACMPL format)
%% Supplied to authors by publisher for camera-ready submission
\acmJournal{PACMPL}
\acmVolume{1}
\acmNumber{1}
\acmArticle{1}
\acmYear{2017}
\acmMonth{1}
\acmDOI{10.1145/nnnnnnn.nnnnnnn}
\startPage{1}
\else\makeatother
%% Conference information (used by SIGPLAN proceedings format)
%% Supplied to authors by publisher for camera-ready submission
\acmConference[PL'17]{ACM SIGPLAN Conference on Programming Languages}{January 01--03, 2017}{New York, NY, USA}
\acmYear{2017}
\acmISBN{978-x-xxxx-xxxx-x/YY/MM}
\acmDOI{10.1145/nnnnnnn.nnnnnnn}
\startPage{1}
\fi


%% Copyright information
%% Supplied to authors (based on authors' rights management selection;
%% see authors.acm.org) by publisher for camera-ready submission
\setcopyright{none}             %% For review submission
%\setcopyright{acmcopyright}
%\setcopyright{acmlicensed}
%\setcopyright{rightsretained}
%\copyrightyear{2017}           %% If different from \acmYear

\bibliographystyle{ACM-Reference-Format}
\citestyle{acmauthoryear}   %% For author/year citations

\begin{document}

\title{Algebraic Graphs with Class}
                                        %% [Short Title] is optional;
                                        %% when present, will be used in
                                        %% header instead of Full Title.
%\titlenote{with title note}
                                        %% \titlenote is optional;
                                        %% can be repeated if necessary;
                                        %% contents suppressed with 'anonymous'
%\subtitle{}
                                        %% \subtitle is optional
%\subtitlenote{with subtitle note}      %% \subtitlenote is optional;
                                        %% can be repeated if necessary;
                                        %% contents suppressed with 'anonymous'

%% Author information
%% Contents and number of authors suppressed with 'anonymous'.
%% Each author should be introduced by \author, followed by
%% \authornote (optional), \orcid (optional), \affiliation, and
%% \email.
%% An author may have multiple affiliations and/or emails; repeat the
%% appropriate command.
%% Many elements are not rendered, but should be provided for metadata
%% extraction tools.

%% Author with single affiliation.
\author{First1 Last1}
\authornote{with author1 note}          %% \authornote is optional;
                                        %% can be repeated if necessary
\orcid{nnnn-nnnn-nnnn-nnnn}             %% \orcid is optional
\affiliation{
  \position{Position1}
  \department{Department1}              %% \department is recommended
  \institution{Institution1}            %% \institution is required
  \streetaddress{Street1 Address1}
  \city{City1}
  \state{State1}
  \postcode{Post-Code1}
  \country{Country1}
}
\email{first1.last1@inst1.edu}          %% \email is recommended

%% Paper note
%% The \thanks command may be used to create a "paper note" ---
%% similar to a title note or an author note, but not explicitly
%% associated with a particular element.  It will appear immediately
%% above the permission/copyright statement.
%\thanks{with paper note}               %% \thanks is optional
                                        %% can be repeated if necesary
                                        %% contents suppressed with 'anonymous'


%% Abstract
%% Note: \begin{abstract}...\end{abstract} environment must come
%% before \maketitle command
\begin{abstract}
The paper presents a minimalistic, elegant and powerful approach to working
with graphs in a functional programming language. The approach is built on
a rigorous mathematical foundation --- an algebra of graphs --- that allows
to apply equational reasoning for proving the correctness of graph transformation
algorithms. One useful feature of the presented approach compared to the
state-of-the-art graph libraries is that it allows to avoid partial functions on
graphs, which are typically required for handling `malformed graphs' that contain an
edge referring to a non-existent vertex. Algebraic graphs make it impossible to
specify such malformed graphs, thereby solving this issue at the root.

The basic definition of algebraic graphs corresponds to unlabelled directed graphs.
We show that by introducing additional axioms to the algebra, it is possible to also
represent undirected, reflexive, transitively closed, and labelled graphs.
We derive basic graph transformation algorithms that are polymorphic and can
therefore be reused by all graph instances, and demonstrate the flexibility of
algebraic graphs by a few examples.
\end{abstract}

%% 2012 ACM Computing Classification System (CSS) concepts
%% Generate at 'http://dl.acm.org/ccs/ccs.cfm'.
\begin{CCSXML}
<ccs2012>
<concept>
<concept_id>10002950.10003624.10003633</concept_id>
<concept_desc>Mathematics of computing~Graph theory</concept_desc>
<concept_significance>500</concept_significance>
</concept>
<concept>
<concept_id>10003752.10010124.10010125.10010127</concept_id>
<concept_desc>Theory of computation~Functional constructs</concept_desc>
<concept_significance>500</concept_significance>
</concept>
</ccs2012>
\end{CCSXML}

\ccsdesc[500]{Mathematics of computing~Graph theory}
\ccsdesc[500]{Theory of computation~Functional constructs}%% End of generated code

\keywords{algebra, graph, polymorphism, functional programming}

%% Note: \maketitle command must come after title commands, author
%% commands, abstract environment, Computing Classification System
%% environment and commands, and keywords command.
\maketitle

\section{Introduction}

Graphs are ubiquitous in computing, yet working with graphs still requires
painfully low-level fiddling with lists of vertices and edges. As a motivating
example, we look at two popular Haskell\footnote{In this paper
we exclusively use Haskell, but the presented approach can be adapted to
other functional programming languages.} graph libraries. The first
one is available from the \textsf{containers} package and is an implementation
of the approach by King and Launchbury~\citeyear{1995_king_graphs}, and
the second one is the \textsf{fgl} library, which is built on the work by
Martin Erwig~\citeyear{2001_erwig_inductive}.

% Focus not on graph algorithms, but more on convenient graph
% specification/transformation language.

% "Unlike traditional approaches
% based on mutable references or node/edge lists, well-formedness of
% the graph structure is ensured statically and reasoning can be done
% with standard functional programming techniques."


\section{Algebraic structure\label{sec-algebra}}

\section{Graphs a la carte\label{sec-a-la-carte}}

\section{Graphs transformations\label{sec-transformations}}

\section{Applications\label{sec-applications}}

\section{Related work\label{sec-related}}

In this section we review existing approaches to working with graphs developed
by the functional programming community.

\begin{itemize}
    \item Tying the knot:
    \item Borrowing imperative algorithms via the State monad.
    \item Data.Graph by King & Launchbury, 1995. \url{https://galois.com/wp-content/uploads/2014/08/pub_JL_StructuringDFSAlgorithms.pdf}.
    Clever tricks exploiting lazy evaluation (Johnsson 1998) \url{https://pdfs.semanticscholar.org/a6ed/4e55f148e0c48445269990102838f7d7abb5.pdf?_ga=1.176470501.1134931652.1487554701}

    \item Inductive Graphs by Martin Erwig, 2001. \url{https://web.engr.oregonstate.edu/~erwig/papers/InductiveGraphs_JFP01.pdf}
    \item Structured Graphs by Oliveira & Cook. \url{https://www.cs.utexas.edu/~wcook/Drafts/2012/graphs.pdf}
    \item An initial-algebra approach to directed acyclic graphs, by J. Gibbons
\end{itemize}


%% Acknowledgments
\begin{acks}                            %% acks environment is optional
                                        %% contents suppressed with 'anonymous'
  %% Commands \grantsponsor{<sponsorID>}{<name>}{<url>} and
  %% \grantnum[<url>]{<sponsorID>}{<number>} should be used to
  %% acknowledge financial support and will be used by metadata
  %% extraction tools.
  This material is based upon work supported by the
  \grantsponsor{GS100000001}{National Science
    Foundation}{http://dx.doi.org/10.13039/100000001} under Grant
  No.~\grantnum{GS100000001}{nnnnnnn} and Grant
  No.~\grantnum{GS100000001}{mmmmmmm}.  Any opinions, findings, and
  conclusions or recommendations expressed in this material are those
  of the author and do not necessarily reflect the views of the
  National Science Foundation.
\end{acks}

\bibliography{publications}

\appendix
\section{Appendix}

Text of appendix \ldots

\end{document}
